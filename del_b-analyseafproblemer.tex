
\subsection{Analyse af problemer ved AVIS' webside}
\label{sec:analyse-af-problemer}

I det til opgaveformuleringen vedlagte bilag D har jeg lokaliseret
gennemg�ende temaer i problempunkterne ved at anvende algoritmen p�
\cite[s. 483]{benyon}. Hermed fik jeg dannet et affinitetsdiagram, som
lettere gav et overbliv over kerneproblemerne p� AVIS' webside.

Diagrammet endte med at best� af to overpunkter: ``uoverskuelighed''
samt ``manglende information''. Problemerne i affinitetsdiagrammet er
ikke listet i prioriteret r�kkef�lge.

\begin{center}
  \begin{tabular}{ | c | p{5cm} | p{6cm} |}
    \multicolumn{3}{c}{\textbf{Uoverskuelighed}} \\

    \hline
    Nr. & Titel & Beskrivelse \\ \hline

    1
    & Hj�lpelinks sv�re at finde 
    & AVIS' webside har en r�kke hj�lpelinks, men de er sv�re at finde,
      da de er placeret nederst p� siden
    \\ \hline

    2
    & Uklar opdeling af information
    & Navigationslinksne i topmenuen har misvisende titler ift. de linkede
      siders indhold.
    \\ \hline

    3
    & Hj�lpetekst til formularer misvisende
    & Fx st�r der, at der skal v�lges postnummer, hvilket blot resulterer i
      endnu et formularfelt, hvori land skal indtastes. Endvidere st�r det ikke
      angivet, at datofelterne p� s�geformularen netop er datofelter.
    \\ \hline

    4
    & Navigation inkonsistent
    & Navigation via billeder resulterer i en side, der repr�senterer indholdet
      af billedet. Dette er dog ikke tilf�ldet p� biloversigten, hvor der linkes
      til forsiden.
    \\ \hline

    5
    & S�gning efter udlejningskontorer forvirrende
    & Medmindre man er meget stedkendt, er der ikke nogen mulighed for at finde
      ud af hvor et udlejningskontor ligger, medmindre det lokaliseres andetsteds.
      Endvidere er der problemer, hvis man vil bestille en bil fra et center uden
      for �bningstid, da der ikke henvises til n�rmeste andet biludlejningskontor.
    \\ \hline

    6
    & Tvivl om brug af knapper ved s�gning
    & Der er to knapper p� s�geformularen p� forsiden: ``Forts�t'' og ``F� et
      tilbud''. Der er tvivl hos brugerne om knappernes brug.
    \\ \hline
  \end{tabular}
\end{center}